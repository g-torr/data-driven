%% This is file `elsarticle-template-1-num.tex',
%%
%% Copyright 2009 Elsevier Ltd
%%
%% This file is part of the 'Elsarticle Bundle'.
%% ---------------------------------------------
%%
%% It may be distributed under the conditions of the LaTeX Project Public
%% License, either version 1.2 of this license or (at your option) any
%% later version.  The latest version of this license is in
%%    http://www.latex-project.org/lppl.txt
%% and version 1.2 or later is part of all distributions of LaTeX
%% version 1999/12/01 or later.
%%
%% Template article for Elsevier's document class `elsarticle'
%% with numbered style bibliographic references
%%
%% $Id: elsarticle-template-1-num.tex 149 2009-10-08 05:01:15Z rishi $
%% $URL: http://lenova.river-valley.com/svn/elsbst/trunk/elsarticle-template-1-num.tex $
%%
\documentclass[preprint,12pt]{elsarticle}
\usepackage{siunitx}
%% Use the option review to obtain double line spacing
%% \documentclass[preprint,review,12pt]{elsarticle}

%% Use the options 1p,twocolumn; 3p; 3p,twocolumn; 5p; or 5p,twocolumn
%% for a journal layout:
%% \documentclass[final,1p,times]{elsarticle}
%% \documentclass[final,1p,times,twocolumn]{elsarticle}
%% \documentclass[final,3p,times]{elsarticle}
%% \documentclass[final,3p,times,twocolumn]{elsarticle}
%% \documentclass[final,5p,times]{elsarticle}
%% \documentclass[final,5p,times,twocolumn]{elsarticle}

%% The graphicx package provides the includegraphics command.
\usepackage{graphicx}
\usepackage{grffile}
%% The amssymb package provides various useful mathematical symbols
\usepackage{amssymb}
\usepackage{amsmath}
%% The amsthm package provides extended theorem environments
 \usepackage{amsthm}
\usepackage{subcaption}
\usepackage{breqn}% to break equation over multiple lines use the environment \begin{\dmath} 
%% The lineno packages adds line numbers. Start line numbering with
%% \begin{linenumbers}, end it with \end{linenumbers}. Or switch it on
%% for the whole article with \linenumbers after \end{frontmatter}.
\usepackage{lineno}

%% natbib.sty is loaded by default. However, natbib options can be
%% provided with \biboptions{...} command. Following options are
%% valid:

%%   round  -  round parentheses are used (default)
%%   square -  square brackets are used   [option]
%%   curly  -  curly braces are used      {option}
%%   angle  -  angle brackets are used    <option>
%%   semicolon  -  multiple citations separated by semi-colon
%%   colon  - same as semicolon, an earlier confusion
%%   comma  -  separated by comma
%%   numbers-  selects numerical citations
%%   super  -  numerical citations as superscripts
%%   sort   -  sorts multiple citations according to order in ref. list
%%   sort&compress   -  like sort, but also compresses numerical citations
%%   compress - compresses without sorting
%%
%% \biboptions{comma,round}

% \biboptions{}

\journal{Research method for theoretical modeling}
\usepackage{mathrsfs}
\usepackage{amssymb,amsmath,amsthm}
\newtheorem{theorem}{Theorem}

\begin{document}

\begin{frontmatter}

%% Title, authors and addresses

\title{Data driven analysis }

%% use the tnoteref command within \title for footnotes;
%% use the tnotetext command for the associated footnote;
%% use the fnref command within \author or \address for footnotes;
%% use the fntext command for the associated footnote;
%% use the corref command within \author for corresponding author footnotes;
%% use the cortext command for the associated footnote;
%% use the ead command for the email address,
%% and the form \ead[url] for the home page:
%%
%% \title{Title\tnoteref{label1}}
%% \tnotetext[label1]{}
%% \author{Name\corref{cor1}\fnref{label2}}
%% \ead{email address}
%% \ead[url]{home page}
%% \fntext[label2]{}
%% \cortext[cor1]{}
%% \address{Address\fnref{label3}}
%% \fntext[label3]{}


%% use optional labels to link authors explicitly to addresses:
%% \author[label1,label2]{<author name>}
%% \address[label1]{<address>}
%% \address[label2]{<address>}

\author{Giuseppe Torrisi}

\address{King's College, London}

\begin{abstract}
The Haemodynamic  responce  of  different brain regions during mental arithmetic tasks is investigated.  
\end{abstract}
\end{frontmatter}
\tableofcontents
%%
%% Start line numbering here if you want
%%
%\linenumbers

%% main text
\section{Introduction}
\paragraph{Brain computer interface (BCI)} are devices that aim to interpret brain activity and produce  digital signal that can be read by  a machine. The main challenge is to online classify the brain signal in a  stable and reliable way. In particular the key point for the development in BCI is to find patterns of brain activity associated to specific tasks. 

In this report I analyse the hemodynamic response to metal arithmetic operation.
\subsection{Measurement mechanism}
Haemoglobin (Hb) is a protein found on red cells, it binds up to four $O_2$ molecules, making it the main  Oxygen carrier in blood. The Hb shows structural changes when bound with oxygen. In this case oxy-Hb  rather than deoxy-Hb is formed. Oxy-Hb and deoxy-Hb have different absorption spectra. Since biological tissure is relatively  transparent to light  in the near-infrared range between \SI{700}{\nano\meter} to \SI{1000}{\nano\meter} \cite{jobsis1977noninvasive},  near-infrared spectroscopy (NIRS)  is  suitable to  estimate  Oxy-Hb and deoxy-Hb concentration \cite{villringer1997non}.  

The analysis of hemodynamic response can be used to measure the activity in different areas of brain \cite{buxton2004modeling}. Measurement at the brain tissue level  \cite{malonek1996interactions} have suggested  that in case of higher brain activity more $O_2$ is required, which causes the decrease of oxy-Hb. However, the blood flow rises  to satisfy the higher need of oxygen. In conclusion, after the early decrease of oxy-Hb, the long term effect is the   increases of haemoglobin oxygenation. The latter effect is exploited in NIRS to asses brain activity during the performance of specific tasks, \textit{e.g.} cognitive\cite{franceschini2003hemodynamic}, visual \cite{herrmann2005near}, and motor \cite{wriessnegger2008spatio}.
However, the temporal resolution of hemodynamic response spans over several seconds, making any analysis of NIRS challenging and problematic in the case of BCI.

\subsection{Protocol}
Data  have been recorded from 8  University  students (three males and five females, all right-handed aged 26.0 $\pm$ 2.8 years. They were asked to perform one digit subtraction from a two digit number (\textit{e.g.} $93-5$). The protocol consists in  performing as many mental operation in 12 second, then  a \SI{28}{\second} rest is given. During rest person is asked to relax and not to move. This protocol is repeated 6 times. 3 or 4 run of this protocol is measured per participant.

The device records 52 channel with a sampling rate of \SI{10}{\hertz}
\subsection{Data selection}
The same dataset has already been used in \cite{bauernfeind2011single,pfurtscheller2010focal} to find pattern in the brain response to mental arithmetic tasks. They identify 3 regions of interest (ROI) each containing 3 channel, called ROI-1,ROI-2, and ROI-3 (see Table \ref{Table:ROI}).They show that  a antagonistic activation patterns occurs, consisting in the rise of  oxy-Hb in ROI-2 and ROI-3 and decrease of oxy-Hb in ROI-1. Therefore in the following  only  the channels corresponding to  ROI-1,ROI-2, and ROI-3 are considered.
\begin{table}
\caption{Channels belonging to the three Region of Interest as defined in \cite{bauernfeind2011single}. In this report we examine the following 9 channels only.}
\centering
\begin{tabular}{c|ccc}
	\hline
	ROI-1	&46	&47	&48\\
	\hline
	ROI-2	&18	&28	&29\\		
	\hline
   	ROI-3	&13	&23	&24\\   
   \end{tabular}   
   \label{Table:ROI}
\end{table}
 The authors point out that oxy-Hb concentration provides stronger and less noisy signal compared to  deoxy-Hb concentration. Therefore in this report it has been chosen to consider the oxy-Hb concentration only. 


\section{Cusum}
Cusum 





%% The Appendices part is started with the command \appendix;
%% appendix sections are then done as  normal sections
%% \appendix

%% \section{}
%% \label{}

%% References
%%
%% Following citation commands can be used in the body text:
%% Usage of \cite is as follows:
%%   \cite{key}          ==>>  [#]
%%   \cite[chap. 2]{key} ==>>  [#, chap. 2]
%%   \citet{key}         ==>>  Author [#]

%% References with bibTeX database:

\bibliographystyle{model1-num-names}
\bibliography{sample.bib}

%% Authors are advised to submit their bibtex database files. They are
%% requested to list a bibtex style file in the manuscript if they do
%% not want to use model1-num-names.bst.

%% References without bibTeX database:

% \begin{thebibliography}{00}

%% \bibitem must have the following form:
%%   \bibitem{key}...
%%

% \bibitem{}

% \end{thebibliography}


\end{document}

%%
%% End of file `elsarticle-template-1-num.tex'.